\documentclass{article}

\usepackage{amsmath}
\usepackage{outlines}

\begin{document}

\begin{outline}
	\1 $c = f\lambda$
	\1 $t' = \frac{l}{c}$, derived from v = m/s
	\1 $\phi_0=\frac{\omega l}{c}=\frac{2\pi fl}{c}=2\pi\frac{l}{\lambda}\quad\mathrm{radians}$
	\1 The concept of transmission lines takes affect when, $\frac{l}{\lambda} \geq 0.01$

	\1 $C = \frac{1}{\sqrt{\mu_0 \varepsilon_0}} = 3 \times 10^8 \, \text{m} \cdot \text{s}^{-1}$

	\1 $V_p = \frac{1}{\sqrt{\mu \varepsilon}} = \frac{1}{\sqrt{\mu_0 \mu_r \varepsilon_0 \varepsilon_r}}$

	For non-magnetic materials or mediums (if material or medium is not stated, it can be assumed to be non-magnetic material or medium), $\mu_r = 1$. If $\mu_r$ is not 1, it'll be stated.

	$= \frac{1}{\sqrt{\mu_0 \varepsilon_0} \sqrt{\varepsilon_r}} = \frac{c}{\sqrt{\varepsilon_r}}$

	\1 $
		V_p
		= \frac{c}{\sqrt{\varepsilon_r}}
		= \frac{f\lambda}{\sqrt{\varepsilon_r}}
	$

	\1 $\lambda = \frac{V_p}{f}$

	\1 Electrical length, $l_e = \beta l$

	\2 Unit: rad
	\2 $l$ is the physical length

	\1 $\beta = \dfrac{2 \pi}{\lambda}$

	\1 $2\pi = 1 \lambda$

	\2 One wavelength is like one cycle of a circle

	\1 $Z_{in} = Z_l, \quad \text{for } l = n \dfrac{\lambda}{2} $

	\1  $Z_{in} = \dfrac{Z_0^2}{Z_L}, \quad \text{for } l = \dfrac{\lambda}{4} + n \dfrac{\lambda}{2} $
\end{outline}

Math

\begin{outline}
	\1 $x \angle \theta = x e^{j\theta}$
\end{outline}

\end{document}
