% Options for packages loaded elsewhere
% Options for packages loaded elsewhere
\PassOptionsToPackage{unicode}{hyperref}
\PassOptionsToPackage{hyphens}{url}
\PassOptionsToPackage{dvipsnames,svgnames,x11names}{xcolor}
%
\documentclass[
  letterpaper,
  DIV=11,
  numbers=noendperiod]{scrartcl}
\usepackage{xcolor}
\usepackage{amsmath,amssymb}
\setcounter{secnumdepth}{-\maxdimen} % remove section numbering
\usepackage{iftex}
\ifPDFTeX
  \usepackage[T1]{fontenc}
  \usepackage[utf8]{inputenc}
  \usepackage{textcomp} % provide euro and other symbols
\else % if luatex or xetex
  \usepackage{unicode-math} % this also loads fontspec
  \defaultfontfeatures{Scale=MatchLowercase}
  \defaultfontfeatures[\rmfamily]{Ligatures=TeX,Scale=1}
\fi
\usepackage{lmodern}
\ifPDFTeX\else
  % xetex/luatex font selection
\fi
% Use upquote if available, for straight quotes in verbatim environments
\IfFileExists{upquote.sty}{\usepackage{upquote}}{}
\IfFileExists{microtype.sty}{% use microtype if available
  \usepackage[]{microtype}
  \UseMicrotypeSet[protrusion]{basicmath} % disable protrusion for tt fonts
}{}
\makeatletter
\@ifundefined{KOMAClassName}{% if non-KOMA class
  \IfFileExists{parskip.sty}{%
    \usepackage{parskip}
  }{% else
    \setlength{\parindent}{0pt}
    \setlength{\parskip}{6pt plus 2pt minus 1pt}}
}{% if KOMA class
  \KOMAoptions{parskip=half}}
\makeatother
% Make \paragraph and \subparagraph free-standing
\makeatletter
\ifx\paragraph\undefined\else
  \let\oldparagraph\paragraph
  \renewcommand{\paragraph}{
    \@ifstar
      \xxxParagraphStar
      \xxxParagraphNoStar
  }
  \newcommand{\xxxParagraphStar}[1]{\oldparagraph*{#1}\mbox{}}
  \newcommand{\xxxParagraphNoStar}[1]{\oldparagraph{#1}\mbox{}}
\fi
\ifx\subparagraph\undefined\else
  \let\oldsubparagraph\subparagraph
  \renewcommand{\subparagraph}{
    \@ifstar
      \xxxSubParagraphStar
      \xxxSubParagraphNoStar
  }
  \newcommand{\xxxSubParagraphStar}[1]{\oldsubparagraph*{#1}\mbox{}}
  \newcommand{\xxxSubParagraphNoStar}[1]{\oldsubparagraph{#1}\mbox{}}
\fi
\makeatother


\usepackage{longtable,booktabs,array}
\usepackage{calc} % for calculating minipage widths
% Correct order of tables after \paragraph or \subparagraph
\usepackage{etoolbox}
\makeatletter
\patchcmd\longtable{\par}{\if@noskipsec\mbox{}\fi\par}{}{}
\makeatother
% Allow footnotes in longtable head/foot
\IfFileExists{footnotehyper.sty}{\usepackage{footnotehyper}}{\usepackage{footnote}}
\makesavenoteenv{longtable}
\usepackage{graphicx}
\makeatletter
\newsavebox\pandoc@box
\newcommand*\pandocbounded[1]{% scales image to fit in text height/width
  \sbox\pandoc@box{#1}%
  \Gscale@div\@tempa{\textheight}{\dimexpr\ht\pandoc@box+\dp\pandoc@box\relax}%
  \Gscale@div\@tempb{\linewidth}{\wd\pandoc@box}%
  \ifdim\@tempb\p@<\@tempa\p@\let\@tempa\@tempb\fi% select the smaller of both
  \ifdim\@tempa\p@<\p@\scalebox{\@tempa}{\usebox\pandoc@box}%
  \else\usebox{\pandoc@box}%
  \fi%
}
% Set default figure placement to htbp
\def\fps@figure{htbp}
\makeatother





\setlength{\emergencystretch}{3em} % prevent overfull lines

\providecommand{\tightlist}{%
  \setlength{\itemsep}{0pt}\setlength{\parskip}{0pt}}



 


\usepackage{pdfpages}
\KOMAoption{captions}{tableheading}
\makeatletter
\@ifpackageloaded{caption}{}{\usepackage{caption}}
\AtBeginDocument{%
\ifdefined\contentsname
  \renewcommand*\contentsname{Table of contents}
\else
  \newcommand\contentsname{Table of contents}
\fi
\ifdefined\listfigurename
  \renewcommand*\listfigurename{List of Figures}
\else
  \newcommand\listfigurename{List of Figures}
\fi
\ifdefined\listtablename
  \renewcommand*\listtablename{List of Tables}
\else
  \newcommand\listtablename{List of Tables}
\fi
\ifdefined\figurename
  \renewcommand*\figurename{Figure}
\else
  \newcommand\figurename{Figure}
\fi
\ifdefined\tablename
  \renewcommand*\tablename{Table}
\else
  \newcommand\tablename{Table}
\fi
}
\@ifpackageloaded{float}{}{\usepackage{float}}
\floatstyle{ruled}
\@ifundefined{c@chapter}{\newfloat{codelisting}{h}{lop}}{\newfloat{codelisting}{h}{lop}[chapter]}
\floatname{codelisting}{Listing}
\newcommand*\listoflistings{\listof{codelisting}{List of Listings}}
\makeatother
\makeatletter
\makeatother
\makeatletter
\@ifpackageloaded{caption}{}{\usepackage{caption}}
\@ifpackageloaded{subcaption}{}{\usepackage{subcaption}}
\makeatother
\usepackage{bookmark}
\IfFileExists{xurl.sty}{\usepackage{xurl}}{} % add URL line breaks if available
\urlstyle{same}
\hypersetup{
  pdftitle={Study guides},
  colorlinks=true,
  linkcolor={blue},
  filecolor={Maroon},
  citecolor={Blue},
  urlcolor={Blue},
  pdfcreator={LaTeX via pandoc}}


\title{Study guides}
\author{}
\date{}
\begin{document}
\maketitle


\subsection{Topic 8 - IPv4 Addressing Study Guide Study
Guide}\label{topic-8---ipv4-addressing-study-guide-study-guide}

\begin{enumerate}
\def\labelenumi{\arabic{enumi}.}
\tightlist
\item
  How many bits are in an IPv4 address, and how are they grouped? (S3)
\end{enumerate}

\begin{itemize}
\tightlist
\item
  Each address contains 32 bits, divided into four octets.
\end{itemize}

\begin{enumerate}
\def\labelenumi{\arabic{enumi}.}
\setcounter{enumi}{1}
\tightlist
\item
  IPv4 addresses have two parts; what are they, and what process is used
  identify them? (S8,9)
\end{enumerate}

\begin{itemize}
\tightlist
\item
  Network portion and a host portion.
\item
  A subnet mask is used to determine the network and host portions in a
  process called ANDing.
\end{itemize}

\begin{enumerate}
\def\labelenumi{\arabic{enumi}.}
\setcounter{enumi}{2}
\tightlist
\item
  What is the prefix length? (S10)
\end{enumerate}

\begin{itemize}
\tightlist
\item
  The number of bits set to 1 in the subnet mask
\item
  It is written in ``slash notation''
\end{itemize}

\begin{enumerate}
\def\labelenumi{\arabic{enumi}.}
\setcounter{enumi}{3}
\tightlist
\item
  What operation is used to identify the network address? (S11)
\end{enumerate}

\begin{itemize}
\tightlist
\item
  To identify the network address, the host IPv4 address is logically
  ANDed, bit by bit, with the subnet mask.
\end{itemize}

\begin{enumerate}
\def\labelenumi{\arabic{enumi}.}
\setcounter{enumi}{4}
\tightlist
\item
  How can you tell if an IPv4 address is a host, network or broadcast
  address? (S6)
\end{enumerate}

\begin{itemize}
\tightlist
\item
  Network address: all host bits set to 0.
\item
  Host addresses: host bits range from (``Network address'' + 1) to
  (``Broadcast address'' - 1)
\item
  Broadcast address: all host bits set to 1.
\end{itemize}

\begin{enumerate}
\def\labelenumi{\arabic{enumi}.}
\setcounter{enumi}{5}
\tightlist
\item
  Define the following types of network addresses. (S13-15)
\end{enumerate}

\begin{enumerate}
\def\labelenumi{\alph{enumi})}
\tightlist
\item
  Unicast

  \begin{itemize}
  \tightlist
  \item
    Sending a packet to one destination IP address.
  \end{itemize}
\item
  Broadcast

  \begin{itemize}
  \tightlist
  \item
    Sending a packet to all IP addresses in the network.
  \end{itemize}
\item
  Multicast

  \begin{itemize}
  \tightlist
  \item
    Sending a packet to a multicast address group.
  \end{itemize}
\end{enumerate}

\begin{enumerate}
\def\labelenumi{\arabic{enumi}.}
\setcounter{enumi}{6}
\tightlist
\item
  Compare the public and private IPv4 addresses. (S16)
\end{enumerate}

\begin{itemize}
\tightlist
\item
  Public IPv4 addresses are globally routed between internet service
  provider (ISP) routers.
\item
  Private addresses are not globally routable.
\end{itemize}

\begin{enumerate}
\def\labelenumi{\arabic{enumi}.}
\setcounter{enumi}{7}
\tightlist
\item
  What are the Private address ranges? (S16)
\end{enumerate}

\begin{longtable}[]{@{}ll@{}}
\toprule\noalign{}
Network Address and Prefix & RFC 1918 Private Address Range \\
\midrule\noalign{}
\endhead
\bottomrule\noalign{}
\endlastfoot
10.0.0.0/8 & 10.0.0.0 - 10.255.255.255 \\
172.16.0.0/12 & 172.16.0.0 - 172.31.255.255 \\
192.168.0.0/16 & 192.168.0.0 - 192.168.255.255 \\
\end{longtable}

\begin{enumerate}
\def\labelenumi{\arabic{enumi}.}
\setcounter{enumi}{8}
\tightlist
\item
  What does NAT stand for, and what is its function? (S17)
\end{enumerate}

\begin{itemize}
\tightlist
\item
  Network Address Translation (NAT) translates private IPv4 addresses to
  public IPv4 addresses.
\end{itemize}

\begin{enumerate}
\def\labelenumi{\arabic{enumi}.}
\setcounter{enumi}{9}
\tightlist
\item
  What are the Loopback addresses and how are they used? (S18)
\end{enumerate}

\begin{itemize}
\tightlist
\item
  127.0.0.0/8 (127.0.0.1 to 127.255.255.254)
\item
  Commonly identified as only 127.0.0.1
\item
  Used on a host to test if TCP/IP is operational.
\end{itemize}

\begin{enumerate}
\def\labelenumi{\arabic{enumi}.}
\setcounter{enumi}{10}
\tightlist
\item
  Complete the following table for the Classful Addressing. (12) (S19)
\end{enumerate}

\begin{longtable}[]{@{}
  >{\raggedright\arraybackslash}p{(\linewidth - 6\tabcolsep) * \real{0.2500}}
  >{\raggedright\arraybackslash}p{(\linewidth - 6\tabcolsep) * \real{0.2500}}
  >{\raggedright\arraybackslash}p{(\linewidth - 6\tabcolsep) * \real{0.2500}}
  >{\raggedright\arraybackslash}p{(\linewidth - 6\tabcolsep) * \real{0.2500}}@{}}
\toprule\noalign{}
\begin{minipage}[b]{\linewidth}\raggedright
Class
\end{minipage} & \begin{minipage}[b]{\linewidth}\raggedright
Default Subnet Mask/Prefix
\end{minipage} & \begin{minipage}[b]{\linewidth}\raggedright
Network Address Range
\end{minipage} & \begin{minipage}[b]{\linewidth}\raggedright
Number ofHosts/Network
\end{minipage} \\
\midrule\noalign{}
\endhead
\bottomrule\noalign{}
\endlastfoot
A & /8 & 0.0.0.0/8 to 127.0.0.0/8 & 16,777,214 \\
B & /16 & 128.0.0.0 /16 - 191.255.0.0 /16 & 65,534 \\
C & /24 & 192.0.0.0 /24 --223.255.255.0 /24 & 254 \\
\end{longtable}

\begin{enumerate}
\def\labelenumi{\arabic{enumi}.}
\setcounter{enumi}{11}
\tightlist
\item
  Compare switches and routers with respect to broadcast propagation.
  (S21)
\end{enumerate}

\begin{itemize}
\tightlist
\item
  Switches propagate broadcasts out all interfaces except the interface
  on which it was received.
\item
  Routers do not propagate broadcasts.
\end{itemize}

\begin{enumerate}
\def\labelenumi{\arabic{enumi}.}
\setcounter{enumi}{12}
\tightlist
\item
  Explain the problem with large broadcast domains and how it can be
  resolved. (S22)
\end{enumerate}

\begin{itemize}
\tightlist
\item
  large broadcast domain is that these hosts can generate excessive
  broadcasts and negatively affect the network.
\item
  The solution is to reduce the size of the network to create smaller
  broadcast domains in a process called subnetting.
\end{itemize}

\begin{enumerate}
\def\labelenumi{\arabic{enumi}.}
\setcounter{enumi}{13}
\tightlist
\item
  List three reasons for segmenting networks. (S23)
\end{enumerate}

\begin{itemize}
\tightlist
\item
  To reduces overall network traffic and improves network performance.
\item
  To implement security policies between subnets.
\item
  To group the users according to location, group/function, or device
  type.
\end{itemize}

\begin{enumerate}
\def\labelenumi{\arabic{enumi}.}
\setcounter{enumi}{14}
\tightlist
\item
  Covert the following numbers: (S5-7)
\end{enumerate}

\begin{itemize}
\tightlist
\item
  10110 (Decimal) into Binary
\item
  101010012 (Binary) into Decimal
\end{itemize}

\begin{enumerate}
\def\labelenumi{\arabic{enumi}.}
\setcounter{enumi}{15}
\tightlist
\item
  Write the subnet mask next to the following addresses. (S10)
\end{enumerate}

\begin{longtable}[]{@{}ll@{}}
\toprule\noalign{}
\endhead
\bottomrule\noalign{}
\endlastfoot
192.168.5.53/22 & \\
192.168.2.44/26 & \\
192.168.2.44/28 & \\
\end{longtable}

\begin{enumerate}
\def\labelenumi{\arabic{enumi}.}
\setcounter{enumi}{16}
\tightlist
\item
  Write the prefix length next to the following subnet masks. (S10)
\end{enumerate}

\begin{longtable}[]{@{}ll@{}}
\toprule\noalign{}
\endhead
\bottomrule\noalign{}
\endlastfoot
255.255.255.128 & \\
255.255.252.0 & \\
255.255.248.0 & \\
\end{longtable}

18.Determine the address types and write the results in column 2.
(S11-12)

\paragraph{Example 1}\label{example-1}

192.168.1.32/27 → Network

\[ 27 = 24 + 3 \]

The first 3 octets are common to host, broadcast, and network addresses.

\[ 32 = 00100000 \] Note: \(32\) is a \emph{network} octet.

\[ 63 = 11100000 \] Note: \(63\) is a \emph{broadcast} octet.

\begin{aligned}
N &= 192.168.1.32 \\
B &= 192.168.1.63
\end{aligned}

\[ \text{First address} = N + 1 = 192.168.1.33 \]
\[ \text{Last address} = B - 1 = 192.168.1.62 \]

\begin{center}\rule{0.5\linewidth}{0.5pt}\end{center}

\paragraph{Example 2}\label{example-2}

200.25.36.200/25

\[ /25 \implies \text{Host bits} = 32 - 25 = 7 \]

The first 3 octets are common to host, broadcast, and network addresses.

\begin{itemize}
\tightlist
\item
  Last octet of:

  \begin{itemize}
  \tightlist
  \item
    given ip address: 200 = 11001000
  \item
    subnet mask: 10000000
  \item
    ANDing a) and b) to get N's octet: \((10000000)_2 = (128)_{ 10 }\)
  \item
    Changing all host bits in c) to 1 to get B's octet:
    \((11111111)_2 = (255)_{ 10 }\)
  \end{itemize}
\end{itemize}

\begin{aligned}
N &= 200.25.36.128 \\
B &= 200.25.36.255
\end{aligned}

The given address is a host address

\begin{center}\rule{0.5\linewidth}{0.5pt}\end{center}

\paragraph{Example 3}\label{example-3}

172.16.55.71/29

\[ /29 \implies \text{Host bits} = 32 - 29 = 3 \]

The first 3 octets are common to host, broadcast, and network addresses.

\begin{itemize}
\tightlist
\item
  Last octet of:

  \begin{itemize}
  \tightlist
  \item
    given ip address: 71 = 01000111
  \item
    subnet mask: 11111000
  \item
    ANDing a) and b) to get N's octet: \((01000000)_2 = (64)_{ 10 }\)
  \item
    Changing all host bits in c) to 1 to get B's octet:
    \((01000111)_2 = (71)_{ 10 }\)
  \end{itemize}
\end{itemize}

\begin{aligned}
N &= 172.16.55.64 \\
B &= 172.16.55.71
\end{aligned}

The given address is a broadcast address

\begin{center}\rule{0.5\linewidth}{0.5pt}\end{center}

\paragraph{Example 4}\label{example-4}

10.2.3.75/28




\end{document}
